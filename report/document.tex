\documentclass[conference]{IEEEtran}
\IEEEoverridecommandlockouts
% The preceding line is only needed to identify funding in the first footnote. If that is unneeded, please comment it out.
\usepackage[british,UKenglish,USenglish,english,american]{babel}

\usepackage{cite}
\usepackage{amsmath,amssymb,amsfonts}
\usepackage{algorithmic}
\usepackage{graphicx}
\usepackage{textcomp}
\usepackage{xcolor}
\def\BibTeX{{\rm B\kern-.05em{\sc i\kern-.025em b}\kern-.08em
    T\kern-.1667em\lower.7ex\hbox{E}\kern-.125emX}}
\begin{document}

\title{Optimization of dynamic routing, space and spectrum allocation
(RSSA) of unicast demands in flex-grid network\\
{\footnotesize \textsuperscript{} Zaawansowane metody projektowania sieci teleinformatycznych}
}

\author{\IEEEauthorblockN{1\textsuperscript{st} Bakowicz Maciej}
\IEEEauthorblockA{\textit{Wrocław University of Science and Technology} \\
Faculty of Electronics \\
Wrocław, Poland \\
}
\and
\IEEEauthorblockN{2\textsuperscript{nd} Czajkowski Michał}
\IEEEauthorblockA{\textit{Wrocław University of Science and Technology} \\
Faculty of Electronics \\
Wrocław, Poland \\
}
}

\maketitle

\begin{abstract}
This document is a model and instructions for \LaTeX.
This and the IEEEtran.cls file define the components of your paper [title, text, heads, etc.]. *CRITICAL: Do Not Use Symbols, Special Characters, Footnotes, 
or Math in Paper Title or Abstract.
\end{abstract}

\begin{IEEEkeywords}
component, formatting, style, styling, insert
\end{IEEEkeywords}

\section{Introduction}
This document is a model and instructions for \LaTeX.
Please observe the conference page limits. 

\section{Related works}
To better understand the problem it is necessary to achieve some basic knowledge about how the SDM networks are built and what solution are used there (e.g. in physical fiber links)\cite{b4}.
\\ \\
To solve the optimization problem we use Python 3.7.1\cite{b1} programming language due to it's flexibility and huge amounts of well developed mathematics and data structures packages. Python is also the language we know the most among the others so we can be sure solving the problem within it is going to go without solving side problems such as language syntax.
\\ \\
For more complex data structures we use Numpy \cite {b2} which is a Python package. It provides structures like matrix, vector or more advanced dictionaries than Python itself. With this package also comes methods for that structures, like deleting, reshaping, transforming or transposing.
\\ \\
In a summary of this project we include some of data plots. To make it easy and simple we use Python side plots and datagrams provided by another package - matplotlib \cite{b3}. Besides simple plots this package allows to comparing different data structures within one plot or just export generated plot into image format such as JPEG or PNG.

\begin{thebibliography}{00}
\bibitem{b1} G. Eason, B. Noble, and I. N. Sneddon, ``On certain integrals of Lipschitz-Hankel type involving products of Bessel functions,'' Phil. Trans. Roy. Soc. London, vol. A247, pp. 529--551, April 1955.
\bibitem{b2} A
\bibitem{b3} B
\bibitem{b4} A. Muhammad, G. Zervas, G. Saridis, E. H. Salas, D. Simeonidou, R. Forchheimer ``Flexible and Synthetic SDM Networks with Multi-core-Fibers Implemented by Programmable ROADMs'', Conference: ECOC 2014, At Cannes -France, September 2014
\end{thebibliography}
\vspace{12pt}
\end{document}
