\documentclass[conference]{IEEEtran}
\IEEEoverridecommandlockouts
% The preceding line is only needed to identify funding in the first footnote. If that is unneeded, please comment it out.
\usepackage[british,UKenglish,USenglish,english,american]{babel}

\usepackage{cite}
\usepackage{float}
\usepackage{amsmath,amssymb,amsfonts}
\usepackage{algorithmic}
\usepackage{graphicx}
\usepackage{textcomp}
\usepackage{csvsimple}
\usepackage{listings}
\usepackage{xcolor}
\def\BibTeX{{\rm B\kern-.05em{\sc i\kern-.025em b}\kern-.08em
    T\kern-.1667em\lower.7ex\hbox{E}\kern-.125emX}}
\begin{document}

\title{Optimization of dynamic routing, space and spectrum allocation
(RSSA) of unicast demands in flex-grid network\\
{\footnotesize \textsuperscript{} Zaawansowane metody projektowania sieci teleinformatycznych}
}

\author{\IEEEauthorblockN{Maciej Bakowicz}
\IEEEauthorblockA{\textit{Wrocław University of Science and Technology} \\
Faculty of Electronics \\
Wrocław, Poland \\
}
\and
\IEEEauthorblockN{Michał Czajkowski}
\IEEEauthorblockA{\textit{Wrocław University of Science and Technology} \\
Faculty of Electronics \\
Wrocław, Poland \\
}
}

\maketitle

\begin{abstract}
This document is about the SDM networks optimization problem. It describes our authorial implementation of optimization algorithm. 
\end{abstract}

\begin{IEEEkeywords}
flex-grid, RSSA, SDM, optimization, algorithms, SMF, optical fibre
\end{IEEEkeywords}

\section{Introduction}
Flex-grid architecture uses a space division technology. As it is going beyond the capabilities of WDM/EON systems, most likely it will be commonly used in networks to increase transmitted capacity within a single-mode fiber (SMF) \cite{sdm-walko}. Using this solution encounters other problem due to non-linear Shannon limit which is related to maximum power which can be applied into an optical fibre (effects such as chromatic dispersion and non-linearity of the signal cause additional signal disortion)\cite{shannon}.
\\
Space Division Multiplexing (SDM) architecture used by optical networks solve the problem with insufficient capacity of bandwidth by redistributing the traffic (or to be more specific a part of it - \textit{slices}) over cores inside a multicore fiber (MCF). Each core inside one fiber can also support different bit rate \cite{flex-intro}.
\\
The hard part of using SDM with MCF is optimizing the traffic to allocate as much bandwidth in each core as possible without data loss or delays and to change the path when allocating cores within a current is impossible.
\\
This document describes a solution for such problem by implementing algorithms in a software (an application).
\\

\section{Related works}
Our optimization problem definition's backbone is \cite{roza} and \cite{differential}. To better understand the problem it is vital to acquire some basic knowledge about how the SDM networks are built and what solutions are used there (e.g. in physical fiber links) \cite{sdm-intro}.
Moreover, further information can be drawn from \cite{rssa2} and \cite{walkoartykul}.
\\ \\
With SDM networks comes the problem of routing space and spectrum allocation (RSSA) inside a network which can be solved by using many methods and algorithms. In \cite{wb-box} the solution about  switching node designs as a extended black box was mentioned. It stands with comparison to the concept of optical white box which is a more flexible and scalable alternative for such networks as SDM. However, black and white box solutions are related more to routing, modulation, spectrum and core allocation (RMSCA) than to RSSA but some concepts and parts of solutions might be applied as a RSSA solutions.
\\
In \cite{sdm-walko} as a solution for RSSA problem the greedy algorithm was proposed. The implementation of this algorithm in a given work is mostly used to analyse available path and spectral-spatial channel (SSCh - a concept of two dimensional structure that represents allocation both in space and spatial domain). In the results the algorithm chooses the most suitable path and channel based on current demand (request, programmable a single iteration within a loop).
\\ \\
According to greedy algorithm, two enhancements can be done in RSSA processing - algorithm parallelization and application of dedicated
data structures\cite{rssa}.
\

\section{Problem definition}
The optimization objective is minimizing demand/bitrate blocking probability. The network, being defined by a set of nodes and physical links implements SDM technology, thus on each physical link there is a given set of fiber cores. Each core consists of 320 segments, i.e. slices, and each slice is 12.5 GHz wide. Channel can be created by grouping particular number of adjacent slices, then it is characterized by first slice's index, number of involved slices and index of applied fiber core.
\\ \\
In this case, routing path has to be a sequence of network links, where the source node is connected with the destination node. Routing path has to be established before data can be sent between path ends (pair of nodes). When the path is established, spectrum resources have to be selected and allocated, the light-path is a connection of routing path with a channel allocated on the path links. It is required in order to realize a demand in an optical network, and, when established, has to connect demand end nodes wherein the channel size is a function of:
\begin{itemize}
\item demand volume in Gbps
\item path length in kilometers
\item modulation format
\end{itemize}
The optimization algorithm has to follow four constraints:
\begin{itemize}
\item spectrum contiguity
\item spectrum continuity
\item spectrum non-overlapping
\item guard-band between neighbouring connections
\end{itemize}
These constraints will be further discussed in upcoming section.
\\
The major goal of this algorithm is to reduce the amount of used slices in total, and be the most spectrum-effective, so in other words we want to allocate as much demands as it is possible. Each of the demands is described by:
\begin{itemize}
\item sourde node
\item destination node
\item bitrate (Gbps)
\item duration (iterations)
\end{itemize}
Iterations will be set to the value of 2000. Dynamic scenario is considered - demands arrival process observed within a specified time period.
We assume that demands arrival rate will follow Poisson process, while their duration follows exponential process.
\\ \\
In the considered problem - dynamic RSSA - we focus on existing network in its operational state. We simulate its work considering a particular time period (iterations in this case). In each iteration, a number of demands is incoming and a number of existing demands is expiring. The new incoming demands need to be, if possible, assigned with a light-path, with respect to the current resource availability, while the expiring demands need to release assigned resources. It is possible that for a particular network state (i.e., resource availability), there is no possibility to allocate a demand (due to lack of enough free slices). Then, that demand is rejected (remains unallocated). Please note that in each iteration it is necessary to keep information about existing demands (and control the time of their existence in the network) as well as allocated/free resources (slices on links and cores). Moreover, at the end of each iteration it is necessary to release (free) resources allocated for demands, whose duration currently ends. The released resources can be used in next iterations by other demands.

\section{Solution algorithm}
Suggested algorithm consists of a few components which are equivalents for real parts in SDM network topology. Respectively:
\begin{itemize}
\item \textbf{Node} - represents physical node within network topology, described by given name.
\item \textbf{Link} - represents connection between pair of nodes and is a part of given paths required for each demand. Described as start and end point (both are node objects) also a link length is a known value. Each link consists of given number of cores. Link objects are programmed to take care of channel allocation for cores within it for demands.
\item \textbf{Core} - represents physical core inside a link (fiber). Described by name (a link id) and iterated id. Each core contains a matrix (1x320) which represents slices and their indexes; 1 for taken index 0 for free. Each of the core objects checks if it can be allocated by given demand slices and based on the result whether to allocate or not space for them. When it's requested core object unallocates space (e.g. when demand is finished).
\item \textbf{Path} - simplified structure containing a key, which is a value combined from source and the destination and a list of links which are continuously connected by nodes (i.e. those links create a valid \textbf{route}).
\item \textbf{Demand} - a physical demand (request). Described as unique id, start (a number of iteration which demand needs to be sent in), the source and destination (both are node objects), bitrate (in Gbps), duration (a number of iterations) and slices (equivalent of real data slices).
\end{itemize}
Algorithm follows four principles:
\begin{itemize}
\item spectrum contiguity - all demand slices have to be in the same fiber core (cannot be slitted between cores in the same fiber)
\item spectrum continuity - demand has to use the same fiber core and the same indexes range within links in the demand path
\item spectrum non-overlapping - a slice in a core cannot be shared between different demand slices
\item guard-band between neighbouring connections - each slices sequence in core has to have additional one slice taken as a guard-band to prevent crosstalk between connections (demands)
\end{itemize}

Basically algorithm is a loop over a given number of iterations (2000). At the beginning of each iteration given list of demands are created based on theirs start and duration. This process is needed to skip those demands which do not require to be considered in the iteration. Basically this list can be split onto two - demands which are going to be finished in the iteration and those which are going to start. Firstly a space should be unallocated, so before starting new demands those which are going to be finished are served. It prevents from choosing different (in this case a longer one) path for a demand.

\subsection{Checking space availability}
At first algorithm checks path availability. If for some reason a path for given demand source and destination does not exists a demand is going to be failed. \ 
When a path/s (usually more than just one) are found they are sorted by their lengths (shortest path in the beginning). Then the space is calculated for a path which was selected. A selected path can be unavailable (due to not enough free slices in cores) so a space availability check for a single demand can take place multiple times. If a available path are not found a demand is going to be failed. Otherwise a space for this demand is allocated. \

\begin{small}
\begin{lstlisting}[frame=single]
for demand in prepared_demands:
  paths_candidates = topology.get_paths(
    demand.source.id,
    demand.destination.id
  )
    
  number_of_paths = len(paths_candidates)
  chosen_path = 0
  if number_of_paths > 0:
    while chosen_path < number_of_paths:
      is_success = demand.check_and_allocate(
        iteration,
        paths_candidates[chosen_path]
      )

      if is_success:
        break
      else:
        chosen_path = chosen_path + 1
  else:
    demand.mark_as_failed()
\end{lstlisting}
\textit{An example code implementation for checking demand and selecting path.}
\end{small}
\\

A space availability check process can be described by few steps:
\begin{itemize}
\item For each link in a selected path a space on given required core for required slice indexes are checked. If there is no restriction for required core and slice indexes a first core is checked (with checking slices from the beginning) - includes one extra slot needed due to spectrum non-overlapping principle. If a a space for demand was found for this link a next one is checked with the same value for required core and slices indexes (which were changed due to previous check). If there is no space on given indexes for a core a core is checked once more for space, excluding already checked. If a space was found for different indexes previous links are checked once more with new required slice indexes. If a core has no space at all for this demand another one is checked (regardless on given required core - if a space for a different core has been found, all links in a path are checked once more with new value for required core). If all possibilities was checked and at least one link cannot be allocated no matter what configuration is a demand is going to check another path (if avaiable). E.g:
  \begin{itemize}
  \item A path P1 was selected for a demand with N-1 slices . For a Link L1 a space on Core C1 was found on indexes [X1,...,XN]
  \item However Link L2 has no space on core C1, but it was found successfully on core C2 with indexes [X5,...,XN-5].
  \item Link L1 (and any other link) are checked with a new configuration of indexes and cores. A space for new requirements was found on other links
  \item A demand is being allocated on core C2 with indexes [X5,...,XN-5]
  \end{itemize}
\end{itemize}

\begin{small}
\begin{lstlisting}[frame=single]
for link in self.path.links:
  r, si, c = link.check_channel_availability(
    self.slices,
    self.required_core_id,
    self.required_start_index,
    self.already_checked_indexes,
    self.already_checked_cores,
  ) 

  if not r:
    self.required_start_index = None
    self.required_core_id = None
    return False

  if self.required_core_id is None:
    self.required_core_id = c.id
  elif self.required_core_id != c.id:
    self.already_checked_indexes = []
    self.already_checked_cores.append(c)
    self.required_core_id = c.id
    break

  if self.required_start_index is None:
    self.required_start_index = si
  elif self.required_start_index != si:
    self.already_checked_indexes.append(
      self.required_start_index
    )
    self.required_start_index = si
    break
    
  checked_links = checked_links + 1

\end{lstlisting}
\end{small}

If for some reason a check was interrupted (not every link was checked) check again with new configuration:
\begin{small}
\begin{lstlisting}[frame=single]
if checked_links != len(self.path.links):
  return self.check_resources()
\end{lstlisting}
\textit{An example code implementation for checking core availability within links.}
\end{small}
\\

\subsection{Allocating space for new demand}
An allocation process takes place when a requirements for a demand are known (a path, required core and required slice indexes).

\subsection{Unallocating space for demand}
When demand requires to be finished, slices related to links in this demand path are being unallocated.

\section{Related software}
\subsection{File parser}
In addition to algorithm implementation a programmable file parser for given data files was implemented.
\\ \\
Based on three files a topology of a network is created. 
\begin{itemize}
\item f30.spec contains number of slices related to bitrate for each path
\item ff30.pat contains connections between pair of nodes (Links)
\item ff.net contains lengths of Links 
\end{itemize}

\noindent Demands objects are created based on fourth file:
\begin{itemize}
\item \textit{net\_load}\_\textit{set\_nr}.dem, where
\begin{itemize}
\item \textit{net\_load} is a average network load in Erlangs
\item \textit{set\_nr} is number of the demands set
\end{itemize}
\end{itemize}

\subsection{Logger}
A short piece of program which implements creating brief summary in a *.csv format for results output such as:
\begin{itemize}
\item used configration (files, number of cores etc)
\item duration of a allocation process for a demand file
\item success / failure ratio for demands
\item number of shortest paths used
\end{itemize}

\section{Algorithm results}
\begin{footnotesize}
\noindent \textit{Results for file 100\_01.dem} \\
\csvautotabular[separator=semicolon]{100_01.dem_summary_fixed_1.csv}\\\\
\csvautotabular[separator=semicolon]{100_01.dem_summary_fixed_3.csv}\\\\
\csvautotabular[separator=semicolon]{100_01.dem_summary_fixed_8.csv}\\\\

\noindent \textit{Results for file 100\_08.dem} \\
\csvautotabular[separator=semicolon]{100_08.dem_summary_fixed_1.csv}\\\\
\csvautotabular[separator=semicolon]{100_08.dem_summary_fixed_3.csv}\\\\
\csvautotabular[separator=semicolon]{100_08.dem_summary_fixed_8.csv}\\\\

\noindent \textit{Results for file 400\_01.dem} \\
\csvautotabular[separator=semicolon]{400_01.dem_summary_fixed_1.csv}\\\\
\csvautotabular[separator=semicolon]{400_01.dem_summary_fixed_3.csv}\\\\
\csvautotabular[separator=semicolon]{400_01.dem_summary_fixed_8.csv}\\\\

\noindent \textit{Results for file 400\_08.dem} \\
\csvautotabular[separator=semicolon]{400_08.dem_summary_fixed_1.csv}\\\\
\csvautotabular[separator=semicolon]{400_08.dem_summary_fixed_3.csv}\\\\
\csvautotabular[separator=semicolon]{400_08.dem_summary_fixed_8.csv}\\\\

\noindent \noindent \textit{Results for file 1000\_01.dem} \\
\csvautotabular[separator=semicolon]{1000_01.dem_summary_fixed_1.csv}\\\\
\csvautotabular[separator=semicolon]{1000_01.dem_summary_fixed_8.csv}\\\\

\noindent \textit{Results for file 1000\_08.dem} \\ 
\csvautotabular[separator=semicolon]{1000_08.dem_summary_fixed_1.csv}\\\\
\csvautotabular[separator=semicolon]{1000_08.dem_summary_fixed_3.csv}\\\\

\noindent \textit{Results for file 2000\_01.dem} \\
\csvautotabular[separator=semicolon]{2000_01.dem_summary_fixed_1.csv}\\\\
\csvautotabular[separator=semicolon]{2000_01.dem_summary_fixed_3.csv}\\\\

\noindent \textit{Results for file 2000\_08.dem} \\
\csvautotabular[separator=semicolon]{2000_08.dem_summary_fixed_1.csv}\\\\
\csvautotabular[separator=semicolon]{2000_08.dem_summary_fixed_3.csv}\\\\
\end{footnotesize}
Columns in these tables are, accordingly: 
\begin{itemize}
\item Erlangs - unit of the network load
\item Total demands - how many demands there were
\item Failed - amount of failed demands
\item Shortest - number of demands which used the shortest path
\item Cores - number of cores used in the simulation
\item Duration - time taken by the simulation
\end{itemize}

As expected the number of failed demands is much higher for those simulations with less core used.\\

The specific feature of our program is that it does not create only one occurrence of path, and accordingly that is why there from time to time appears to be the same amount of fails. 

\section{Tools used}
To solve the optimization problem we use Python 3.6.0 \cite{python} programming language due to it's flexibility, ease of use and huge amount of well developed mathematics and data structures packages. Python is also the language we know the most among the others so we can be sure solving the problem within it is going to go without solving side problems such as language syntax.
\\ \\
For more complex data structures we use Numpy \cite{numpy} which is a Python package. It provides structures like matrix, vector or more advanced dictionaries than Python itself. With this package also comes methods for that structures, like deleting, reshaping, transforming or transposing.
\\ \\
In a summary of this project we include some of data plots. To make it easy and simple we use Python side plots and datagrams provided by another package - matplotlib \cite{matplotlib}. Besides simple plots this package allows comparison of different data structures within one plot or just export generated plot into image format such as JPEG or PNG.
\\

\begin{thebibliography}{00}

\bibitem{sdm-walko} P. Lechowicz, K. Walkowiak, M. Klinkowski ``Selection of Spectral-Spatial Channels in SDM
Flexgrid Optical Networks'', Department of Systems and Computer Networks, Wrocław University of Science and Technology, National Institute of Telecommunications, 1 Szachowa Street, 04-894 Warsaw, Poland, 2017

\bibitem{shannon} A.D. Ellis ``The nonlinear Shannon limit and the need for new fibres'', Page 2-3, Tyndall National Institute \& Dept. of Physics, University College Cork, Dyke Parade, Cork, Ireland

\bibitem{flex-intro} D. Rafique, T. Rahman, A Napoli, M. Kuschnerov, G. Lehmann, B. Spinnler ``Flex-grid optical networks: spectrum allocation and nonlinear dynamics of super-channels'', Page 1-2, Coriant R\&D GmbH, St.-Martin-Str. 76, 81541, Munich, Germany, Eindhoven University of Technology, Eindhoven, Netherlands, 2013

\bibitem{roza} R. Goścień ``Dynamic routing, space and spectrum allocation (RSSA) of unicast demands in flex-grid network'', Wroclaw University of Science and Technology, Wroclaw, Poland

\bibitem{differential} F. Lezama, G. Castanon, A. M. Sarmiento, I. B. Martins, ``Routing and Spectrum Allocation in Flexgrid Optical Networks
Using Differential Evolution Optimization'', Department of Electrical and Computer Engineering, 2Department of Industrial Engineering,
Tecnológico de Monterrey, Ave. Eugenio Garza Sada \#2501 Sur, Monterrey NL, 64849, México

\bibitem{sdm-intro} A. Muhammad, G. Zervas, G. Saridis, E. H. Salas, D. Simeonidou, R. Forchheimer ``Flexible and Synthetic SDM Networks with Multi-core-Fibers Implemented by Programmable ROADMs'', Conference: ECOC 2014, At Cannes -France, September 2014

\bibitem{rssa2} D. Siracusa, F. Pederzolli, D. Klonidis, V. Lopez, E. Salvadori ''Resource Allocation Policies in
SDM Optical Networks'',  in Proc. of ONDM, Pisa, Italy, May 2015, pp. 168–173.

\bibitem{walkoartykul} M. Klinkowski, P. Lechowicz, K. Walkowiak ``Survey of Resource Allocation Schemes and Algorithms in Spectrally-Spatially Flexible Optical Networking'', National Institute of Telecommunications, 1 Szachowa Street, 04-894 Warsaw, Department of Systems and Computer Networks, Wrocław University of Science and Technology, Poland, September 2017

\bibitem{wb-box} A. Muhammad, G. Zervas, R. Forchheimer ``Resource Allocation for Space Division Multiplexing: Optical White Box vs. Optical Black Box Networking '', Page 2-3, Linköping University, Linköping, Sweden, High-Performance Networks Group, University of Bristol, UK

\bibitem{rssa} M. Klinkowski, G. Zalewski, K. Walkowiak ``Optimization of Spectrally and Spatially Flexible Optical Networks with Spatial Mode Conversion'', in Proc. of ONDM, Dublin, Ireland, May 2018, pp. 148-153.

\bibitem{python} Python Software Foundation ``Python 3.7.1 documentation'', access via the Internet: \textit{https://docs.python.org/3/} from 16th November 2018.

\bibitem{numpy} The Scipy community ``NumPy Reference'', June 10, 2017, acces via the Internet \textit{https://docs.scipy.org/doc/numpy-1.13.0/reference/} from 16 November 2018

\bibitem{matplotlib} J. Hunter, D. Dale, E. Firing, M. Droettboom ``Matplotlib Release 3.0.0'', September 21, 2018, access via the Internet \textit{https://matplotlib.org/Matplotlib.pdf}


\end{thebibliography}
\vspace{12pt}
\end{document}
